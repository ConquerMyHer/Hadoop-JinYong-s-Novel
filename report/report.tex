\documentclass[a4paper,UTF8]{article}

\usepackage[margin=1.25in]{geometry}
\usepackage{color}
\usepackage{graphicx}
\usepackage{amssymb}
\usepackage{amsmath}
\usepackage{amsthm}
\usepackage{enumerate}
\usepackage{bm}
\usepackage{hyperref}
\usepackage{epsfig}
\usepackage{color}
\usepackage{mdframed}
\usepackage{lipsum}
\usepackage{mathtools}
\usepackage{algorithm}
\usepackage{algorithmic}
\usepackage{listings}
\usepackage{xcolor}
\usepackage{float}
\usepackage{caption}
\usepackage{mathrsfs}
\usepackage{amsmath}
\usepackage[utf8]{inputenc}
\usepackage[UTF8]{ctex}

\newmdtheoremenv{thm-box}{myThm}
\newmdtheoremenv{prop-box}{Proposition}
\newmdtheoremenv{def-box}{define}

\setlength{\evensidemargin}{.25in}
\setlength{\textwidth}{6in}
\setlength{\topmargin}{-0.5in}
\setlength{\topmargin}{-0.5in}

\usepackage{indentfirst}
\setlength{\parindent}{2em}

\usepackage{subfigure}
% \setlength{\textheight}{9.5in}
%%%%%%%%%%%%%%%%%%set header and footer here%%%%%%%%%%%%%%%%%%
\usepackage{fancyhdr}
\usepackage{lastpage}
\usepackage{layout}
\footskip = 10pt
\pagestyle{fancy}
\lhead{2020, Spring}
\chead{大数据综合处理实验}
\rhead{金庸的江湖——金庸武侠小说中的人物关系挖掘}
\cfoot{\thepage}
\renewcommand{\headrulewidth}{1pt}  			%header
\setlength{\skip\footins}{0.5cm}    			
\renewcommand{\footrulewidth}{0pt}  		

\makeatletter 							
\def\headrule{{\if@fancyplain\let\headrulewidth\plainheadrulewidth\fi%
\hrule\@height 1.0pt \@width\headwidth\vskip1pt	
\hrule\@height 0.5pt\@width\headwidth  			
\vskip-2\headrulewidth\vskip-1pt}      			
 \vspace{6mm}}     						
\makeatother

\graphicspath{{img/}}

\lstset{
 columns=fixed,
 basicstyle = \footnotesize,
 breakatwhitespace=false,         % 设置是否当且仅当在空白处自动中断.
 breaklines=true,
 numbers=left,                                        % 在左侧显示行号
 numberstyle=\tiny\color{gray},                       % 设定行号格式
 frame=none,                                          % 不显示背景边框
 backgroundcolor=\color[RGB]{245,245,244},            % 设定背景颜色
 keywordstyle=\color[RGB]{40,40,255},                 % 设定关键字颜色
 numberstyle=\footnotesize\color[RGB]{96,96,96},
 commentstyle=\color[RGB]{0,128,0},                % 设置代码注释的格式
 stringstyle=\rmfamily\slshape\color[RGB]{128,0,0},   % 设置字符串格式
 showstringspaces=false,                              % 不显示字符串中的空格
 language=JAVA,
 extendedchars=true,
 escapeinside=''                                       % 设置语言
}

%%%%%%%%%%%%%%%%%%%%%%%%%%%%%%%%%%%%%%%%%%%%%%
\numberwithin{equation}{section}
\newtheorem{myThm}{myThm}
\newtheorem*{myDef}{Definition}
\newtheorem*{mySol}{Solution}
\newtheorem*{myProof}{Proof}
\newcommand{\indep}{\rotatebox[origin=c]{90}{$\models$}}
\newcommand*\diff{\mathop{}\!\mathrm{d}}

\graphicspath{{img/}}

\usepackage{multirow}
\renewcommand\refname{reference}
\author{组长:韩畅,组员:李展烁、王一之、闫旭芃}
\begin{document}
%\listoffigures
\captionsetup[figure]{labelfont={bf},labelformat={default},labelsep=period,name={图}}
\title{金庸的江湖——金庸武侠小说中的人物关系挖掘}
\maketitle

\section{实验规划与设计}
\subsection{任务分配}
{171860551, \text{韩畅:组长,算法设计与实验规划,任务一、任务六,程序试运行与组织debug研讨}}\\ \indent
{171860550, \text{王一之:算法设计与实验规划,任务四,参与debug,实验版本控制}}\\ \indent
{171860549, \text{闫旭芃:算法设计与实验规划,任务五,参与debug与数据核对}}\\ \indent
{171840565, \text{李展烁:算法设计与实验规划,任务一优化、任务二、任务三,参与debug并提出重要优化思路}}
\subsection{任务要求}

\subsection{设计思路}


\section{实验实现}
\subsection{任务一:数据预处理}
\subsubsection{设计思路}
数据输入:已经分词、分段的多篇中文文本文件
\par 数据输出:每一段或几段中包含的所有人名,按顺序依次输出。
\par 需要注意的是,由于我们的实验是对金庸全部人物关系的分析,
因此无需特别地注意不同文件及文件名。
\par 我们使用提供的名单列表,通过HashSet结构,快速地比对词语
是否位于名单列表当中。
\par 在装载名单列表时,由于姓名数量的有限性,且由于
configuration只能传输字符串,
因此使用一个私有的将其全部装载到
一个字符串中,填入configuration中,
并在map的setup阶段将其提取出来,
并存储于HashSet中。
\par 特别地,为了提高效率,注意使用StringBuilder和String的相互配合。

\subsubsection{程序分析}
Main 主函数类:
\par 为了传递名单列表,调用NameLoader类的load方法,
从输入的文件名中获取名单,并使用configuration进行配置以备后用。
其余大多为路径、类名的配置,不赘述。如图\ref{task1-Main}
\begin{figure}[htbp]
    \centering
    \includegraphics[width = 15cm]{task1-Main.png}
    \caption{主函数类具体实现}
    \label{task1-Main}
\end{figure}
\par NameLoader类:
\par 使用FileInputStream获取文件内容,使用BufferReader进行缓冲区文件读入操作。
\par 使用StringBuilder快速从单行读入中获取姓名,并快速构建字符串。如图\ref{task1-NameLoader}
\begin{figure}[htbp]
    \centering
    \includegraphics[width = 15cm]{task1-NameLoader.png}
    \caption{名单装载器具体实现}
    \label{task1-NameLoader}
\end{figure}

\par Mapper类:
\par 在setup中从configuration中获取之前装载的名单字符串,经过分词处理后,存入HashSet的名单表中。
\par 在map的重载函数中,对每行也即每个value值以空格为分隔符分割,并依此使用HashSet的查找操作比对是否属于姓名。并将姓名归并为一个字符串输出。如图\ref{task1-mapper}
\begin{figure}[htbp]
    \centering
    \includegraphics[width = 15cm]{task1-mapper.png}
    \caption{mapper具体实现}
    \label{task1-mapper}
\end{figure}

\par Reducer类:
\par 不做操作,直接将获取的值输出即可。

\subsection{任务二}
todo

\subsection{任务三}
todo

\subsection{任务四:基于人物关系图的PageRank计算}
\subsubsection{PageRank算法介绍}
PageRank,又称网页排名,名字源于google创始人之一的Larry Page,是Google公司所使用的对与网页重要性排序的算法。\\
PageRank通过网页之间的超链接评价网页重要性,它的基本思想是:\\
\begin{enumerate}[1)]
    \item 如果一个网页被多个网页所指向,则该网页比较重要
    \item 如果一个重要的网页指向另一个网页,则另一个网页也比较重要
\end{enumerate}
该算法模拟一个上网者,随机打开一个网页,之后随机点击该网页的链接,统计上网者分布在每个网页的概率。\\
最初,每个网页的概率均等,每次跳转时,网页X将其PR(PageRank)均分到所指向的所有页面,记链接数为L(X),
于是,经过一次跳转后:\\
$$
PR(A)=\frac{PR(B)}{L(B)}+\frac{PR(C)}{L(C)}+\frac{PR(D)}{L(D)}+...
$$
我们将每个网页抽象成一个节点,超链接抽象为有向边,共同构成一个图。
则每次跳转可视为所有页面PR构成的特征向量R与该图的出度邻接矩阵M相乘,即:\\
$$R=
\begin{bmatrix}
    PR(p_1) \\
    PR(p_2)\\
    \vdots\\
    PR(p_n)\\
\end{bmatrix}
M=
\begin{bmatrix}
    p_1 \rightarrow p_1 & p_2 \rightarrow p_1 & \cdots & p_n \rightarrow p_1\\
    p_1 \rightarrow p_2 & p_2 \rightarrow p_2 & \cdots & p_n \rightarrow p_2\\
    \vdots & \vdots & \ddots & \vdots\\
    p_1 \rightarrow p_n & p_2 \rightarrow p_n & \cdots & p_n \rightarrow p_n\\
\end{bmatrix}
$$
\\
$$
R_1=M R_0
$$
多次迭代后,PR值趋于稳定,即为最终的PR值。

\subsubsection{设计思路}
任务四的输入为任务三的输出,格式如下:\\
人物 [名字$_1$,影响$_1$|名字$_2$,影响$_2$|...|名字$_n$,影响$_n$] \\
影响$_i$ 为 名字$_i$ 与该人物归一化后的同现次数,表示 名字$_i$ 对该人物的影响权重。\\
每个人物视为图的一个节点,边权重为二人同现次数。\\
对于普通的PageRank计算,由于会存在自环边以及无出度的节点,为方式到达某一节点后陷入该点,会加入“随机浏览者”(random surfer)的概念,
即到达某个节点后有一定概率直接跳转到任意一个节点,从而避免此情况。然而在此次任务中,首先没有自身与自身同现的情况,因此无自旋边;
同时A与B同现,则B一定与A也同现,因此不考虑权重时所有边实际都为无向边,因此不存在出度为0的节点。所以此次任务无需引入“随机浏览者”。\\

MapReduce框架下,运算分布进行,因此不使用邻接矩阵,而采用邻接表的形式。算法大致分为三阶段:\\

阶段一:预处理
首先要将输入格式化为供之后迭代处理的形式。采用如下格式:
key:人物\\
value:PageRank\#[名字$_1$,影响$_1$|名字$_2$,影响$_2$|...|名字$_n$,影响$_n$] \\
以概率为初始值,PageRank应设置为1/N,但N值较大,较小数字做乘法时误差较大,因此将初始PR设置为1来减小误差。\\

阶段二:迭代计算
迭代计算PR值,直到PR收敛。\\
在Mapper中,首先输出如下键值对:
key:人物\\
value:\#出度表\\
此对目的在于维护出度表,value前加\#使reducer便于区分。\\
之后计算PR值,记A的出度表集合为N,则计算过程如下式:
$$
NewPR(A)=\sum_{x\in N}OldPR(x)*weight(x\rightarrow A)    
$$
计算得到新的PR值,再输出一组键值对:
key:人物\\
value:新PageRank值\\
在Reducer中,首先查看value前是否有\#号以区分该键值对类型。
由于是一个迭代过程,将输出格式化,与Mapper的输入格式相同。\\

阶段三:处理结果
在Mapper阶段去除结果中的出度表,只保留PR值。\\
利用Partition类进行排序,由于默认为升序,结果需要降序,因此重写DoubleWritable。
因为只有1000余数据,未采用采样排序,使用了简单的全排序。\\
Reducer阶段整理输出即可

\subsubsection{代码讲解}
程序可分为三个模块:PageRank、RageResultSort以及调度模块。
模块一:PageRank\\
此模块包含了阶段一与阶段二。
\begin{figure}[H]
    \centering

    \includegraphics[width = 15cm]{PageRankMain.png}

    \caption{PageRankMain}
\end{figure}
main函数中配置指定程序运行15次,每次的结果储存在以运行次数为名的文件夹内,下一次迭代的输入为上一次的输出。

\begin{figure}[H]
    \centering

    \includegraphics[width = 15cm]{PageRankMapper.png}

    \caption{PageRankReducer}
\end{figure}
Mapper阶段,首先
\begin{figure}[H]
    \centering

    \includegraphics[width = 15cm]{PageRankReducer.png}

    \caption{PageRankReducer}
\end{figure}


\subsection{任务五}
todo

\subsection{任务六:基于PageRank的可视化}
\subsubsection{设计思路}
金庸的全部武侠小说中总共包含人物一千余名,其存在的相互关联可能达到100K的等级,
在这样庞杂的人物关系中,想要让全部的人物和关系呈现在受众面前,是不现实的。
这一判断得到了很多前车之鉴的证实,例如使用gephi软件生成的关系图,如同一团乱麻,
几乎看不清任何的姓名或关系,更枉谈“得到一些有趣的结论”。
\par 在这样的前提下,我们选择了对可视化的内容进行取舍,将可视化的人物数量级从1000降低
到了100的量级。同时对人物的重要性、人物关系的亲密度以PageRank及其排序的结果
进行刻画,从而得到了比较好的视觉效果。
\par 为了进行个性化的开发,我们选择了Qt作为可视化部分的开发平台,使用C++作为开发语言,
编写了一个RelationPainter的程序,对数据进行个性化的、延拓性强的可视化操作。
利用该程序,我们可以
\begin{itemize}
    \item 通过文件装载按钮,使用txt文件作为输入,直接获得效果图;
    \item 通过关系曲线的粗细更直观地表示了人物关系的亲密程度;
    \item 通过人物姓名和标识点的尺寸,可以更直观地感受人物的重要性;
    \item 通过关系曲线的颜色,更清晰地辨别同一个人所具有的人物关系;
    \item 通过关系曲线的高亮,更快地辨别某一特定个体的关系网;
    \item 通过操控“显示最重要的n位人物”拖动条,控制显示在视野中的人数;
    \item 通过操控“调整显示尺寸”拖动条,控制人物关系显示的疏密程度,从而避免人物的重叠;
    \item 通过将人物名称和人物标识点围成圆环状,避免了姓名与人物关系线条的重叠。
\end{itemize}
该程序的上限很高,延拓性很广,我们将在后面相关章节中详细介绍其改进和优化思路。

\subsubsection{程序分析}
Relation 类:
\par 这是一个刻画单个人物及与之相关的人物关系的类。
其中,通过QList<QPair<QString,double>>
维护了一个人物关系表,从而可以对于该人物相关的
人物关系进行操作。如图\ref{task6-Relation}
\begin{figure}[htbp]
    \centering
    \includegraphics[width = 15cm]{task6-Relation.png}
    \caption{单人物关系类具体实现}
    \label{task6-Relation}
\end{figure}

\par sigFig 类:
\par 这是一个以单人物关系为基础的单图元类。
在这个类中,维护了一个Relation类对象以存储人物关系。
同时根据其中图元绘制参数常量的值,在绘制之前更新其图元的
各项属性(如粗细、颜色、大小、角度、位置、透明度等),
之后通过其中的绘制函数分别绘制人物标志点、人物名和人物关系曲线。
一般来说,人物的PageRank值越高,人物名和标志点绘制得
就越大越粗,而人物关系的权重越高,人物关系曲线就绘制得越粗。
\par 特别的,由于我们将人物标志点和人物名
绘制成圆盘状以避免重叠,而每个人物又在圆盘上
占据不同的角度值,因此需要给每个人物图元在绘制前计算
其在圆盘所占的角度。又因为每个单图元类没有其他图元的角度信息,
因此在绘制关系曲线的时候,需要在更高层的类计算好
角度对应表,并传入关系曲线的绘制函数。

\par 由于类中大部分的方法为更新参数、获取
或设置变量的方法,故不做展示,只集中展示
genAll(更新所有参数),paintDot(绘制标志点),
和paintLine(绘制标志曲线)三个函数的实现。

\begin{figure}[htbp]
    \centering
    \includegraphics[width = 15cm]{task6-sigFig-genAll.png}
    \caption{单图元,全参数更新函数具体实现}
    \label{task6-sigFig-genAll}
\end{figure}
在图\ref{task6-sigFig-genAll}中,我们看到,
在对r(圆盘半径),midX、midY(屏幕中心X、y),mxRk(最大的PageRank值)
设置过后,分别对CirWid(标志点粗细)、TxtWid(人物名粗细)、
DotR(标志点大小)、Deg(标志点角度)、DotXY(标志点坐标)、
TxtXY(人物名坐标)进行了更新。
\begin{figure}[htbp]
    \centering
    \includegraphics[width = 15cm]{task6-sigFig-PaintDot.png}
    \caption{单图元,绘制标志点函数具体实现}
    \label{task6-sigFig-PaintDot}
\end{figure}
在图\ref{task6-sigFig-PaintDot}可以看到,在更新了全部的绘制
参数后,使用Qpainter、QPen、QFont等变量对画笔进行了设置,
将画笔移动一定的角度和坐标,并依此绘制圆和Text文本。
\begin{figure}[htbp]
    \centering
    \includegraphics[width = 15cm]{task6-sigFig-PaintLine.png}
    \caption{单图元,绘制关系曲线函数具体实现}
    \label{task6-sigFig-PaintLine}
\end{figure}
在图\ref{task6-sigFig-PaintLine}中,首先判断是否需要设置高亮色,
并根据之前更新的参数设置画笔,此时注意,QPen的颜色设置内多了一个透明度,
由于我们是通过绘制点集的方式绘制的曲线,因此透明度需要设置得足够低
才有效果(当透明度低时,可以略去很多无效的曲线关系信息)。
特别注意到,传入的degList是一个其他图元的角度表,通过对此表
的查询,才能正确获得关系曲线的目标点位置,从而使用我们自己实现
的简单贝塞尔曲线正确生成关系曲线。同时注意到,图元本身角度和目标点
角度的比较,可以确认目标点与本身的排名先后,与限制位的比较,
可以判断本身及目标点是否处于需要绘制的点集内,从而判断是否需要绘制曲线。

\par sigFigList 类:
\par 作为一个统筹所有单图元的类,其主要的行为是对单图元的参数
在宏观上进行调控,例如控制显示个数、疏密程度、分发颜色、更新角度表
等操作。如图\ref{task6-sigFigList}
\begin{figure}[htbp]
    \centering
    \includegraphics[width = 15cm]{task6-sigFigList.png}
    \caption{分发颜色、更新角度列表、更新疏密程度具体实现}
    \label{task6-sigFigList}
\end{figure}

\par mainwindow 类:
\par 对画布上产生的各类信号给出对应的槽进行处理,具体的任务交给
fgLs去做。如载入文件、设置显示个数以及疏密程度的滚动条、鼠标移动
设置高亮的判断等。如图\ref{task6-mainwindow}
\begin{figure}[htbp]
    \centering
    \includegraphics[width = 15cm]{task6-mainwindow.png}
    \caption{mainwindow具体实现}
    \label{task6-mainwindow}
\end{figure}

\subsubsection{结果展示}
\begin{figure}[htbp]
    \centering
    \includegraphics[width = 15cm]{task6-result.png}
    \caption{可视化结果}
    \label{task6-result}
\end{figure}
打开程序,导入文件并调整滚动条至合适位置,
可以看到最重要的n位人物及其关系列表出现在屏幕
中央。几条最粗的曲线彰显了郭靖与黄蓉、杨过与小龙女、
胡斐与程灵素、慕容复与王语嫣、张无忌与周芷若
等等人物之间的密切关系。通过不同的颜色,可以相对容易地查找与同一人物相关的人物关系。

\section{优化与改进}
\subsection{任务一}
todo

\subsection{任务二}
todo

\subsection{任务三}
todo

\subsection{任务四}
原本迭代次数为20次,在检查中间结果时发现在14次之后,结果变化不大,基本收敛,因此将迭代次数改为15次,节省开销。
\subsection{任务五}
todo

\subsection{任务六}
todo
\section{实验经验总结与改进方向}
\begin{enumerate}[1)]
    \item todo
    \item todo
    \item todo
    \item todo
    \item todo
    \item todo
    \item 在任务六中,由于QPainter的绘制特性缘故,因此人物的姓名有一半是倒着的,造成观感不佳,需要解决。
    \item 在任务六中,由于曲线使用点集绘制,而单点的绘制被自动设置为方形,因此曲线的形状不佳,需要改进。
    \item 在任务六中,由于曲线使用的颜色的亮度、深浅不同,造成人物关系密切程度的直观性降低,需要改进颜色组。
    \item 在任务六中,采用数据结构效率较低,对于冗余的判断和循环没有进行优化,造成操作上的延迟,需要优化。
    \item 在任务六中,由于采用的是高Rank值的人物向低Rank值的人物绘制曲线,因此造成低rank值人物
    快速无法快速辨认全部与其相关的人物关系。可以采用颜色的渐变进行优化。
    \item 在任务六中,尚存在很多绘制参数没有在外部留出接口,例如透明度无法在用户界面设置,需要改进。
\end{enumerate}
\bibliographystyle{plain}
\bibliography{ref}

\end{document}
